\section{Related Work}
\label{sec:related-work}

STEAK falls into the growing body of work relating to HCI in information security.  Barriers
to using secure email include lack of technical aptitude~\cite{why-jonny-cant-encrypt}
~\cite{garfinkel-email-survey}, and
stigmatization of chronic use~\cite{crypto-adoption-criteria}.

There have been several attempts to create usably secure email.
The simplest approach is incorporating key management functions into the 
UIs of existing email clients, as done by Enigmail~\cite{enigmail}.
More sophisticated variants of this approach perform automatic decryption
and verification, and recent variants like Mailpile~\cite{mailpile} attempt
to streamline key distribution and key fingerprint verification.  However, they do not
succeed in making the process transparent---the user is still involved in
trusting, verifying, generating, distributing, and revoking keys.

A more sophisticated approach is to automate key generation and distribution,
as seen in systems like ESCAPE~\cite{escape} and ePOST~\cite{epost}.
In these systems, a user's host generates a key and one or more trusted CAs
automatically vouch for it, allowing users to bootstrap trust in one another.
However, this approach assumes the CAs are never compromised, and requires them
to decide when to revoke keys.
In contrast, STEAK trusts CAs only for registration and backwards
compatibility, and revokes keys when users believe 
their passwords are compromised.

Another approach is ID-based cryptography, whereby system components
derive a user's public key from her identity~\cite{id-based-cryptography}.  This 
eliminates the need for PKI and public key distribution.  However,
revoking unexpired keys is infeasible---either the user must change 
her identity, or the master private key (and all other public keys) must be 
regenerated.

We consider STEAK's architecture to be an incremental improvement on
the Internet Mail 2000~\cite{im2k} proposal.  We take advantage of cloud storage 
independent of the user's ISP, while also addressing security and ubiquitous
access in addition to disincentivizing spam.
